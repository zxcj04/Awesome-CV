%-------------------------------------------------------------------------------
%	SECTION TITLE
%-------------------------------------------------------------------------------
\cvsection{Project Experience}


%-------------------------------------------------------------------------------
%	CONTENT
%-------------------------------------------------------------------------------
\begin{cventries}

%---------------------------------------------------------
  \cventry
    {碩士研究論文} % Job title
    {以機器學習方法估計臺灣遠洋漁業之全球漁獲高解析度分佈} % Organization
    {海大先進計算實驗室} % Location
    {Sep. 2022 - Aug. 2024} % Date(s)
    {
      \begin{cvitems} % Description(s) of tasks/responsibilities
        \item {運用機器學習技術提升臺灣遠洋漁業資源評估準確性}
        \item {結合船舶監控系統(VMS)與電子漁獲日誌,提高捕撈定位準確性及全球漁業熱點分析}
        \item {利用 Python 處理億級漁船歷史軌跡資料,辨識漁船作業區域}
        \item {前端採用 Vue3 結合 CesiumJS,建立高解析度三維 GIS 系統,視覺化漁船作業分佈}
        \item {後端使用 Flask,搭配 Gunicorn 與 Docker,部署於 Ubuntu server,透過 nginx 反向代理處理流量}
        \item {分析結果儲存於 MongoDB 並進行資料索引優化,確保資料快速查詢與使用}
      \end{cvitems}
    }

%---------------------------------------------------------
  \cventry
    {獨力開發專案} % Job title
    {工讀生打卡系統} % Organization
    {海大先進計算實驗室} % Location
    {May. 2023 - Jul. 2023} % Date(s)
    {
      \begin{cvitems} % Description(s) of tasks/responsibilities
        \item{系統設計強調 RWD(響應式網頁設計),適應手機、平板與電腦等裝置}
        \item{前端採用 Vue3 與 Vuetify 建構友善 UI,運用 Vue Router 讓頁面導覽流暢}
        \item{後端使用 Flask 與 Gunicorn 確保效能與穩定性}
        \item{使用 Docker 進行容器化,確保開發與部署環境一致性,快速維護與部署}
        \item{系統部署於 Ubuntu server,透過 nginx 反向代理確保流量管理與安全性}
        \item{使用 MongoDB 優化資料索引,提升使用者資料及打卡記錄查詢效能}
        \item{採用 RedisDB 儲存 auth session,提升驗證速度}
        \item{使用 Git 進行版本控制,建立多分支流程,確保專案有序開發與更新}
      \end{cvitems}
    }

%---------------------------------------------------------
  \cventry
    {Side Project} % Job title
    {MortalGPT 個人 AI 助理} % Organization
    {個人專案} % Location
    {Mar. 2023 - Jun. 2023} % Date(s)
    {
      \begin{cvitems} % Description(s) of tasks/responsibilities
        \item{設計並開發基於 ChatGPT 與 python-telegram-bot 的 Telegram 聊天機器人}
        \item{整合 OpenAI ChatGPT 自然語言處理能力,提供即時互動及流暢的對話體驗}
        \item{利用 python-telegram-bot 處理訊息收發、事件觸發與 Telegram API 整合,簡化開發流程}
        \item{解決多執行緒同步問題,提升機器人回應速度與正確性}
        \item{運用 Python、網路應用開發及 API 整合技術,展示強大的技術整合與問題解決能力}
      \end{cvitems}
    }

%---------------------------------------------------------
\end{cventries}
